\documentclass{article}
% pdf 设置区
%---------------------------------纸张大小设置---------------------------------%
\usepackage{geometry}
\geometry{a4paper,left=1.91cm,right=1.91cm,top=2.05cm,bottom=2.05cm}
%------------------------------------------------------------------------------%


%----------------------------------必要库支持----------------------------------%
\usepackage{xcolor}
\usepackage{tikz}
\usepackage{layouts}
\usepackage{clrscode}
%------------------------------------------------------------------------------%


%--------------------------------添加书签超链接--------------------------------%
\usepackage[unicode=true,colorlinks=false,pdfborder={0 0 0}]{hyperref}
    % 在此处修改打开文件操作
    \hypersetup{
        bookmarks=true,         % show bookmarks bar?
        bookmarksopen=true,     % expanded all bookmark?
        pdftoolbar=true,        % show Acrobat’s toolbar?
        pdfmenubar=true,        % show Acrobat’s menu?
        pdffitwindow=true,      % window fit to page when opened
        pdfstartview={FitH},    % fits the width of the page to the window
        pdfnewwindow=true,      % links in new PDF window
    }
    % 在此处添加文章基础信息
    \hypersetup{
        pdftitle={Math Symbols in LaTeX Manual},
        pdfauthor={polossk},
        pdfsubject={LaTeX Manual},
        pdfcreator={XeLaTeX},
        pdfproducer={XeLaTeX},
        pdfkeywords={LaTeX, Math Symbols},
    }
%------------------------------------------------------------------------------%


%------------------------------添加插图与表格控制------------------------------%
\usepackage{graphicx}
\usepackage[font=small,labelsep=quad]{caption}
\usepackage{booktabs}
\usepackage{tabularx}
\usepackage{setspace}
%------------------------------------------------------------------------------%


%---------------------------------添加列表控制---------------------------------%
\usepackage{enumerate}
\usepackage{enumitem}
%------------------------------------------------------------------------------%


%---------------------------------设置页眉页脚---------------------------------%
\usepackage{fancyhdr}
\usepackage{fancyref}
\pagestyle{fancy}
% \chead{{\fNWPU 西北工业大学} \fSong {\title}}
\lhead{}
\rhead{}
\chead{}
\lfoot{Version: v\artversion}
\cfoot{\thepage}
\rfoot{by \href{https://github.com/polossk}{polossk}}
\renewcommand{\headrulewidth}{0.4pt}
\renewcommand{\headwidth}{\textwidth}
\renewcommand{\footrulewidth}{0pt}
%------------------------------------------------------------------------------%


%-------------------------------数学特殊符号控制-------------------------------%
\usepackage{math-symbols}
%------------------------------------------------------------------------------%


%---------------------------------添加代码控制---------------------------------%
\colorlet{lightgray}{gray!40}
\colorlet{darkred}{red!70!black}
\usepackage{listings}
\lstset{
    basicstyle=\color{darkred}\normalsize\ttfamily,
    backgroundcolor=\color{lightgray},
    breaklines=true,
}
%------------------------------------------------------------------------------%

\endinput
%这是简单的 article 的导言区设置,不能单独编译。
%------------------------------------------------------------------------------%
\newcommand\artversion{2.1.0.0402}
\title{Math-Symbols-in-\LaTeX{}-Manual}
\author{polossk}
\date{Version: v\artversion, Last Update: \today}
%------------------------------------------------------------------------------%
\begin{document}
%------------------------------------------------------------------------------%
% \pagewiselinenumbers
\maketitle

Insert \lstinline`\usepackage{math-symbols}` in your document's preamble.

\tableofcontents
\thispagestyle{fancy}
\renewcommand{\baselinestretch}{1.25}
%------------------------------------------------------------------------------%
\section{Constants and Useful Symbols}
\begin{tabular}{*{10}{l}}
$\mi$ & \lstinline`\mi` & $\mnatr$ & \lstinline`\mnatr` & $\mcmpx$ & \lstinline`\mcmpx` & $\mscab$ & \lstinline`\mscab` & $\mslbg[{[a, b]}]{m}$ & \lstinline`\mslbg[{[a, b]}]{m}`\\
$\mj$ & \lstinline`\mj` & $\mintg$ & \lstinline`\mintg` & $\mhilb$ & \lstinline`\mhilb` & $\mscon{(I)}$ & \lstinline`\mscon{(I)}` & $\mssbl[{[a, b]}]{m}$ & \lstinline`\mssbl[{[a, b]}]{m}`\\
$\me$ & \lstinline`\me` & $\mrato$ & \lstinline`\mrato` & $\mcond$ & \lstinline`\mcond` & $\mslbg{2}$ & \lstinline`\mslbg{2}` & \\
 &  & $\mreal$ & \lstinline`\mreal` & $\mconst$ & \lstinline`\mconst` & $\mssbl{2}$ & \lstinline`\mssbl{2}` & \\
\end{tabular}

\section{Vector and Matrix Defination}
\subsection{Vector Notations}
\begin{tabular}{*{12}{l}}
$\mva$ & \lstinline`\mva` & $\mvj$ & \lstinline`\mvj` & $\mvs$ & \lstinline`\mvs` & $\mvalpha$ & \lstinline`\mvalpha` & $\mvkappa$ & \lstinline`\mvkappa` & $\mvupsilon$ & \lstinline`\mvupsilon`\\
$\mvb$ & \lstinline`\mvb` & $\mvk$ & \lstinline`\mvk` & $\mvt$ & \lstinline`\mvt` & $\mvbeta$ & \lstinline`\mvbeta` & $\mvlambda$ & \lstinline`\mvlambda` & $\mvphi$ & \lstinline`\mvphi`\\
$\mvc$ & \lstinline`\mvc` & $\mvl$ & \lstinline`\mvl` & $\mvu$ & \lstinline`\mvu` & $\mvgamma$ & \lstinline`\mvgamma` & $\mvmu$ & \lstinline`\mvmu` & $\mvchi$ & \lstinline`\mvchi`\\
$\mvd$ & \lstinline`\mvd` & $\mvm$ & \lstinline`\mvm` & $\mvv$ & \lstinline`\mvv` & $\mvdelta$ & \lstinline`\mvdelta` & $\mvnu$ & \lstinline`\mvnu` & $\mvpsi$ & \lstinline`\mvpsi`\\
$\mve$ & \lstinline`\mve` & $\mvn$ & \lstinline`\mvn` & $\mvw$ & \lstinline`\mvw` & $\mvepsilon$ & \lstinline`\mvepsilon` & $\mvxi$ & \lstinline`\mvxi` & $\mvomega$ & \lstinline`\mvomega`\\
$\mvf$ & \lstinline`\mvf` & $\mvo$ & \lstinline`\mvo` & $\mvx$ & \lstinline`\mvx` & $\mvzeta$ & \lstinline`\mvzeta` & $\mvpi$ & \lstinline`\mvpi` & \\
$\mvg$ & \lstinline`\mvg` & $\mvp$ & \lstinline`\mvp` & $\mvy$ & \lstinline`\mvy` & $\mveta$ & \lstinline`\mveta` & $\mvrho$ & \lstinline`\mvrho` & \\
$\mvh$ & \lstinline`\mvh` & $\mvq$ & \lstinline`\mvq` & $\mvz$ & \lstinline`\mvz` & $\mvtheta$ & \lstinline`\mvtheta` & $\mvsigma$ & \lstinline`\mvsigma` & \\
$\mvi$ & \lstinline`\mvi` & $\mvr$ & \lstinline`\mvr` &  &  & $\mviota$ & \lstinline`\mviota` & $\mvtau$ & \lstinline`\mvtau` & \\
\end{tabular}

\subsection{Matrix/Tensor Notations}

Use \lstinline`\mm<name>` or \lstinline`\mt<name>` as the abbr of \underline{M}atrix/\underline{T}ensor.

\begin{tabular}{*{14}{l}}
$\mma$ & \lstinline`\mma` & $\mmg$ & \lstinline`\mmg` & $\mmm$ & \lstinline`\mmm` & $\mms$ & \lstinline`\mms` & $\mmy$ & \lstinline`\mmy` & $\mmgamma$ & \lstinline`\mmgamma` & $\mmsigma$ & \lstinline`\mmsigma`\\
$\mmb$ & \lstinline`\mmb` & $\mmh$ & \lstinline`\mmh` & $\mmn$ & \lstinline`\mmn` & $\mmt$ & \lstinline`\mmt` & $\mmz$ & \lstinline`\mmz` & $\mmdelta$ & \lstinline`\mmdelta` & $\mmupsilon$ & \lstinline`\mmupsilon`\\
$\mmc$ & \lstinline`\mmc` & $\mmi$ & \lstinline`\mmi` & $\mmo$ & \lstinline`\mmo` & $\mmu$ & \lstinline`\mmu` &  &  & $\mmtheta$ & \lstinline`\mmtheta` & $\mmphi$ & \lstinline`\mmphi`\\
$\mmd$ & \lstinline`\mmd` & $\mmj$ & \lstinline`\mmj` & $\mmp$ & \lstinline`\mmp` & $\mmv$ & \lstinline`\mmv` &  &  & $\mmlambda$ & \lstinline`\mmlambda` & $\mmpsi$ & \lstinline`\mmpsi`\\
$\mme$ & \lstinline`\mme` & $\mmk$ & \lstinline`\mmk` & $\mmq$ & \lstinline`\mmq` & $\mmw$ & \lstinline`\mmw` &  &  & $\mmxi$ & \lstinline`\mmxi` & $\mmomega$ & \lstinline`\mmomega`\\
$\mmf$ & \lstinline`\mmf` & $\mml$ & \lstinline`\mml` & $\mmr$ & \lstinline`\mmr` & $\mmx$ & \lstinline`\mmx` &  &  & $\mmpi$ & \lstinline`\mmpi` & \\
\end{tabular}

\subsection{Transposed Matrix Notations}
\begin{tabular}{*{12}{l}}
$\mmat$ & \lstinline`\mmat` & $\mmht$ & \lstinline`\mmht` & $\mmot$ & \lstinline`\mmot` & $\mmvt$ & \lstinline`\mmvt` & $\mmgammat$ & \lstinline`\mmgammat` & $\mmupsilont$ & \lstinline`\mmupsilont`\\
$\mmbt$ & \lstinline`\mmbt` & $\mmit$ & \lstinline`\mmit` & $\mmpt$ & \lstinline`\mmpt` & $\mmwt$ & \lstinline`\mmwt` & $\mmdeltat$ & \lstinline`\mmdeltat` & $\mmphit$ & \lstinline`\mmphit`\\
$\mmct$ & \lstinline`\mmct` & $\mmjt$ & \lstinline`\mmjt` & $\mmqt$ & \lstinline`\mmqt` & $\mmxt$ & \lstinline`\mmxt` & $\mmthetat$ & \lstinline`\mmthetat` & $\mmpsit$ & \lstinline`\mmpsit`\\
$\mmdt$ & \lstinline`\mmdt` & $\mmkt$ & \lstinline`\mmkt` & $\mmrt$ & \lstinline`\mmrt` & $\mmyt$ & \lstinline`\mmyt` & $\mmlambdat$ & \lstinline`\mmlambdat` & $\mmomegat$ & \lstinline`\mmomegat`\\
$\mmet$ & \lstinline`\mmet` & $\mmlt$ & \lstinline`\mmlt` & $\mmst$ & \lstinline`\mmst` & $\mmzt$ & \lstinline`\mmzt` & $\mmxit$ & \lstinline`\mmxit` & \\
$\mmft$ & \lstinline`\mmft` & $\mmmt$ & \lstinline`\mmmt` & $\mmtt$ & \lstinline`\mmtt` &  &  & $\mmpit$ & \lstinline`\mmpit` & \\
$\mmgt$ & \lstinline`\mmgt` & $\mmnt$ & \lstinline`\mmnt` & $\mmut$ & \lstinline`\mmut` &  &  & $\mmsigmat$ & \lstinline`\mmsigmat` & \\
\end{tabular}


\subsection{Special Vector and Matrix Notations}
\begin{tabular}{*{8}{l}}
$\mvzero$ & \lstinline`\mvzero` & $\mvone$ & \lstinline`\mvone` & $\mmzero$ & \lstinline`\mmzero` & $\mmone$ & \lstinline`\mmone` \\
\end{tabular}


\section{Useful Functions and Operators}
\begin{tabular}{*{12}{l}}
$\diff$ & \lstinline`\diff` & $\diag$ & \lstinline`\diag` & $\lcm$ & \lstinline`\lcm` & $\var$ & \lstinline`\var` & $\argmin$ & \lstinline`\argmin` & $\card$ & \lstinline`\card`\\
$\Diff$ & \lstinline`\Diff` & $\eig$ & \lstinline`\eig` & $\rand$ & \lstinline`\rand` & $\corr$ & \lstinline`\corr` & $\argmax$ & \lstinline`\argmax` & $\dist$ & \lstinline`\dist`\\
$\Expect$ & \lstinline`\Expect` & $\tr$ & \lstinline`\tr` & $\mean$ & \lstinline`\mean` & $\conv$ & \lstinline`\conv` & $\argopt$ & \lstinline`\argopt` & \\
\end{tabular}



\section{Useful Aliases and Generators}
\begin{itemize}
\item \lstinline`\fracdiff{}{}`: frac \& diff operator, also provide \lstinline`\dfracdiff{}{}` mode. For example, \lstinline`\fracdiff{u}{x}` gets $\fracdiff{u}{x}$, \lstinline`\dfracdiff{^2u}{x^2}` gets $\dfracdiff{^2u}{x^2}$

\item \lstinline`\fracdiffs{}`: special frac \& diff operator. For example, \lstinline`\fracdiffs{x}` gets $\fracdiffs{x}$, \lstinline`\dfracdiffs{y}` gets $\dfracdiffs{y}$

\item \lstinline`\fracpartial{}{}`: frac \& partial operator, also provide \lstinline`\dfracpartial{}{}` mode. For example, \lstinline`\fracpartial{u}{x}` gets $\fracpartial{u}{x}$, \lstinline`\dfracpartial{^2u}{x^2}` gets $\dfracpartial{^2u}{x^2}$

\item \lstinline`\fracpartials{}`: special frac \& partial operator. For example, \lstinline`\fracpartials{x}` gets $\fracpartials{x}$, \lstinline`\dfracpartials{y}` gets $\dfracpartials{y}$

\item \lstinline`\mclosure{}`, \lstinline`\mclosuresquare{}`, \lstinline`\mclosurebrace{}`: auto height brackets, eg $\mclosurebrace{ \mclosuresquare{ \mclosure{a^2 + b^2}^2 }^2 }$

\item \lstinline`\mfwhen{}{}`: create a symbol \lstinline`|`, eg \lstinline`\mfwhen{\fracpartial{u}{t}}{x=5}` gets $\mfwhen{\fracpartial{u}{t}}{x=5}$

\item \lstinline`\mvct{}{}`, \lstinline`\mvctz{}{}`: row vector creator, eg \lstinline`\mvct{a}{n}` gets $\mvct{a}{n}$, \lstinline`\mvctz{a}{n}` gets $\mvctz{a}{n}$

\item \lstinline`\mvctt{}{}`, \lstinline`\mvctzt{}{}`: column vector creator, eg \lstinline`\mvctt{a}{n}` gets $\mvctt{a}{n}$, \lstinline`\mvctzt{a}{n}` gets $\mvctzt{a}{n}$

\item \lstinline`\mequlist{}`: provided a list of equations, eg \lstinline`\mequlist{x + y &= 10 \\ 4x + 2y &= 30}` gets $\mequlist{x + y &= 10 \\ 4x + 2y &= 30}$, also provide environment \lstinline`equlist`, which is similar with the \lstinline`cases` environment
\end{itemize}

%------------------------------------------------------------------------------%
% \end{multicols}
\end{document}