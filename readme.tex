\documentclass{article}
% pdf 设置区
%---------------------------------纸张大小设置---------------------------------%
\usepackage{geometry}
\geometry{a4paper,left=1.91cm,right=1.91cm,top=2.05cm,bottom=2.05cm}
%------------------------------------------------------------------------------%


%----------------------------------必要库支持----------------------------------%
\usepackage{xcolor}
\usepackage{tikz}
\usepackage{layouts}
\usepackage{clrscode}
%------------------------------------------------------------------------------%


%--------------------------------添加书签超链接--------------------------------%
\usepackage[unicode=true,colorlinks=false,pdfborder={0 0 0}]{hyperref}
    % 在此处修改打开文件操作
    \hypersetup{
        bookmarks=true,         % show bookmarks bar?
        bookmarksopen=true,     % expanded all bookmark?
        pdftoolbar=true,        % show Acrobat’s toolbar?
        pdfmenubar=true,        % show Acrobat’s menu?
        pdffitwindow=true,      % window fit to page when opened
        pdfstartview={FitH},    % fits the width of the page to the window
        pdfnewwindow=true,      % links in new PDF window
    }
    % 在此处添加文章基础信息
    \hypersetup{
        pdftitle={Math Symbols in LaTeX Manual},
        pdfauthor={polossk},
        pdfsubject={LaTeX Manual},
        pdfcreator={XeLaTeX},
        pdfproducer={XeLaTeX},
        pdfkeywords={LaTeX, Math Symbols},
    }
%------------------------------------------------------------------------------%


%------------------------------添加插图与表格控制------------------------------%
\usepackage{graphicx}
\usepackage[font=small,labelsep=quad]{caption}
\usepackage{booktabs}
\usepackage{tabularx}
\usepackage{setspace}
%------------------------------------------------------------------------------%


%---------------------------------添加列表控制---------------------------------%
\usepackage{enumerate}
\usepackage{enumitem}
%------------------------------------------------------------------------------%


%---------------------------------设置页眉页脚---------------------------------%
\usepackage{fancyhdr}
\usepackage{fancyref}
\pagestyle{fancy}
% \chead{{\fNWPU 西北工业大学} \fSong {\title}}
\lhead{}
\rhead{}
\chead{}
\lfoot{Version: v\artversion}
\cfoot{\thepage}
\rfoot{by \href{https://github.com/polossk}{polossk}}
\renewcommand{\headrulewidth}{0.4pt}
\renewcommand{\headwidth}{\textwidth}
\renewcommand{\footrulewidth}{0pt}
%------------------------------------------------------------------------------%


%-------------------------------数学特殊符号控制-------------------------------%
\usepackage{math-symbols}
%------------------------------------------------------------------------------%


%---------------------------------添加代码控制---------------------------------%
\colorlet{lightgray}{gray!40}
\colorlet{darkred}{red!70!black}
\usepackage{listings}
\lstset{
    basicstyle=\color{darkred}\normalsize\ttfamily,
    backgroundcolor=\color{lightgray},
    breaklines=true,
}
%------------------------------------------------------------------------------%

\endinput
%这是简单的 article 的导言区设置,不能单独编译。
\usepackage[normalem]{ulem}
\usepackage{lineno}
%------------------------------------------------------------------------------%
\title{Math-Symbols-in-\LaTeX{}-Manual}
\author{polossk}
\date{Last Update \today}
\newcommand\artversion{1.2.4}
%-Code-Environment-------------------------------------------------------------%
\colorlet{lightgray}{gray!40}
\colorlet{darkred}{red!60!black}
\newcommand\hl%
  {\bgroup\markoverwith{\textcolor{lightgray}{\rule[-.7ex]{2pt}{2.7ex}}}\ULon}
\newcommand\mtex[1]%
  {\hl{\thinspace\color{darkred}{\tt \textbackslash#1}\thinspace}}
\newcommand\code[1]%
  {\hl{\thinspace\color{darkred}{\tt #1}\thinspace}}
%------------------------------------------------------------------------------%
\begin{document}
%------------------------------------------------------------------------------%
\pagewiselinenumbers
\maketitle
\tableofcontents
\thispagestyle{fancy}
\renewcommand{\baselinestretch}{1.25}
\sWuhao\fSong
%------------------------------------------------------------------------------%
\section{Constants and Useful Symbols}
\begin{itemize}
\item \mtex{mi}: alias of \mtex{mathrm i}, $\mi$
\item \mtex{me}: alias of \mtex{mathrm e}, $\me$
\item \mtex{mnatr}: alias of \mtex{mathbb N}, $\mnatr$
\item \mtex{mintg}: alias of \mtex{mathbb Z}, $\mintg$
\item \mtex{mrato}: alias of \mtex{mathbb Q}, $\mrato$
\item \mtex{mreal}: alias of \mtex{mathbb R}, $\mreal$
\item \mtex{mcmpx}: alias of \mtex{mathbb C}, $\mcmpx$
\item \mtex{mhilb}: alias of \mtex{mathbb H}, $\mhilb$
\item \mtex{mcond}: alias of \mtex{mathrm \{Cond.\}}, $\mcond$
\item \mtex{mconst}: alias of \mtex{mathrm \{const\}}, $\mconst$
\item \mtex{mscon\{\}}: continuous function space. eg: \mtex{mscon\{(I)\}} gets $\mscon{(I)}$
\item \mtex{mscab}: continuous function space, alias of \mtex{mscon\{[a, b]\}}, $\mscab$
\item \mtex{mslbg[]\{\}}: lebesgue function space. eg: \mtex{mslbg\{2\}} gets $\mslbg{2}$, \mtex{mslbg[\{[a, b]\}]\{2\}} gets $\mslbg[{[a, b]}]{2}$
\item \mtex{mssbl[]\{\}}: sobolev function space. eg: \mtex{mssbl\{m\}} gets $\mssbl{m}$, \mtex{mssbl[\{[a, b]\}]\{m\}} gets $\mssbl[{[a, b]}]{m}$
\end{itemize}

\section{Vector and Matrix Defination}
\begin{itemize}
\item \mtex{mv*}: Vector Notations, alias of \mtex{bm *}, \code{*} could be any English characters or Greek characters. For examples, \mtex{mva} gets $\mva$, and \mtex{mvalpha} gets $\mvalpha$. The alphabet looks like this: $\mva$, $\mvb$, $\mvc$, $\mvd$, $\mve$, $\mvf$, $\mvg$, $\mvh$, $\mvi$, $\mvj$, $\mvk$, $\mvl$, $\mvm$, $\mvn$, $\mvo$, $\mvp$, $\mvq$, $\mvr$, $\mvs$, $\mvt$, $\mvu$, $\mvv$, $\mvw$, $\mvx$, $\mvy$, $\mvz$, $\mvalpha$, $\mvbeta$, $\mvgamma$, $\mvdelta$, $\mvepsilon$, $\mvzeta$, $\mveta$, $\mvtheta$, $\mviota$, $\mvkappa$, $\mvlambda$, $\mvmu$, $\mvnu$, $\mvxi$, $\mvpi$, $\mvrho$, $\mvsigma$, $\mvtau$, $\mvupsilon$, $\mvphi$, $\mvchi$, $\mvpsi$, $\mvomega$
\item \mtex{mm*}: Matrix Notations, alias of \mtex{mathbf *}, \code{*} could be any English characters or Greek characters. For examples, \mtex{mma} gets $\mma$, and \mtex{mmsigma} gets $\mmsigma$. The alphabet looks like this: $\mma$, $\mmb$, $\mmc$, $\mmd$, $\mme$, $\mmf$, $\mmg$, $\mmh$, $\mmi$, $\mmj$, $\mmk$, $\mml$, $\mmm$, $\mmn$, $\mmo$, $\mmp$, $\mmq$, $\mmr$, $\mms$, $\mmt$, $\mmu$, $\mmv$, $\mmw$, $\mmx$, $\mmy$, $\mmz$, $\mmgamma$, $\mmdelta$, $\mmtheta$, $\mmlambda$, $\mmxi$, $\mmpi$, $\mmsigma$, $\mmupsilon$, $\mmphi$, $\mmpsi$, $\mmomega$
\item \mtex{mm*t}: Transposed Matrix Notations, alias of \mtex{\{\textbackslash{}mathbf *\}\^{}T}, \code{*} could be any English characters or Greek characters. For examples, \mtex{mma} gets $\mma$, and \mtex{mmsigma} gets $\mmsigma$. The alphabet looks like this: $\mmat$, $\mmbt$, $\mmct$, $\mmdt$, $\mmet$, $\mmft$, $\mmgt$, $\mmht$, $\mmit$, $\mmjt$, $\mmkt$, $\mmlt$, $\mmmt$, $\mmnt$, $\mmot$, $\mmpt$, $\mmqt$, $\mmrt$, $\mmst$, $\mmtt$, $\mmut$, $\mmvt$, $\mmwt$, $\mmxt$, $\mmyt$, $\mmzt$, $\mmgammat$, $\mmdeltat$, $\mmthetat$, $\mmlambdat$, $\mmxit$, $\mmpit$, $\mmsigmat$, $\mmupsilont$, $\mmphit$, $\mmpsit$, $\mmomegat$
\item \mtex{mvzero}, \mtex{mvone}, \mtex{mmzero}, \mtex{mmone}: Special vector and matrix notation, $\mvzero$, $\mvone$, $\mmzero$, $\mmone$
\end{itemize}

\section{Useful Functions and Operators}
\begin{itemize}
\item \mtex{diff}: diff operator, $\diff\ $, eg. $\int_0^t f(\tau) \diff \tau$
\item \mtex{Diff}: Diff operator, $\Diff\ $, eg. $\Diff{}^2 X = \dfrac{-x_{i + 1, j} + 2x_{i, j} - x_{i - 1, j}}{\Delta x^2}$
\item \mtex{Expect}: Expect operator, $\Expect$, eg. $X = B(n, p)$, $\Expect X = np$
\item \mtex{diag}, \mtex{eig}, \mtex{tr}: Matrix operators, $\diag$, $\eig$, $\tr$, eg. $\mmd = \diag \mma$, $[\mmlambda, \mmv] = \eig \mma$, $\tr \mmlambda = \tr \mma$
\item \mtex{lcm}: lcm operator, $\lcm$, eg. $\lcm(f, g) \cdot \gcd(f, g) = f \cdot g$
\item \mtex{rand}: random number, $\rand$
\item \mtex{mean}, \mtex{var}: statistics operator, $\mean$, $\var$, eg. $\mu = \mean X$, $\sigma^2 = \var X$
\item \mtex{corr}: correlation operator, $\corr$, eg. $\corr(X, Y) = (R)_{ij} = \dfrac{\sum_{X_i, Y_j}(X - \bar X)(Y - \bar Y)}{\sqrt{\sum_i(X - \bar X)^2\sum_j(Y - \bar Y)^2}}$
\item \mtex{conv}: convolution operator, $\conv$, eg. $\conv(f, g) = \int_{-\infty}^\infty f(\tau)g(t - \tau) \diff \tau$
\item \mtex{card}: cardinals operator, $\card$, eg. $\card \{1, 2, 3\} = 3$, $\card \mreal = 2^{\aleph_0}$
\item \mtex{argmin}, \mtex{argmax}, \mtex{argopt}: $\argmin$, $\argmax$, $\argopt$ operator, $\hat \theta = \argmin\limits_\theta J_\theta(x)$
\item \mtex{dist}: distance operator, $\dist$, eg. $\min \sum\limits_{\forall s, t \in G} \dist\limits_{s \ne t} (s, t)$
\item \mtex{abs\{\}}, \mtex{norm\{\}}: norm operator, eg. $\abs{x+y} \leq \abs{x} + \abs{y}$, $\norm{\mma \mvx + \mvb}$
\item \mtex{normlp\{\}\{\}}: Lp-norm operator, eg. $\normlp{1}{2}$, $\normlp{\mma \mvx + \mvb}{2}$, $\normlp{\mma \mvx + \mvb}{\infty}$
\end{itemize}

\section{Useful Aliases and Generators}
\begin{itemize}

\item \mtex{fracdiff\{\}\{\}}: frac \& diff operator, also provide \mtex{dfracdiff\{\}\{\}} mode. For example, \mtex{fracdiff\{u\}\{x\}} gets $\fracdiff{u}{x}$, \mtex{dfracdiff\{\^{}2u\}\{x\^{}2\}} gets $\dfracdiff{^2u}{x^2}$

\item \mtex{fracdiffs\{\}}: simple frac \& diff operator. For example, \mtex{fracdiffs\{x\}} gets $\fracdiffs{x}$, \mtex{dfracdiffs\{y\}} gets $\dfracdiffs{y}$

\item \mtex{fracpartial\{\}\{\}}: frac \& partial operator, also provide \mtex{dfracpartial\{\}\{\}} mode. For example, \mtex{fracpartial\{u\}\{x\}} gets $\fracpartial{u}{x}$, \mtex{dfracpartial\{\^{}2u\}\{x\^{}2\}} gets $\dfracpartial{^2u}{x^2}$

\item \mtex{fracpartials\{\}}: simple frac \& partial operator. For example, \mtex{fracpartials\{x\}} gets $\fracpartials{x}$, \mtex{dfracpartials\{y\}} gets $\dfracpartials{y}$

\item \mtex{mclosure{}}, \mtex{mclosuresquare{}}, \mtex{mclosurebrace{}}: auto height brackets, eg $\mclosurebrace{ \mclosuresquare{ \mclosure{a^2 + b^2}^2 }^2 }$

\item \mtex{mvct{}{}}, \mtex{mvctz{}{}}: row vector creator, eg \mtex{mvct\{a\}\{n\}} gets $\mvct{a}{n}$, \mtex{mvctz\{a\}\{n\}} gets $\mvctz{a}{n}$

\item \mtex{mvctt{}{}}, \mtex{mvctzt{}{}}: column vector creator, eg \mtex{mvctt\{a\}\{n\}} gets $\mvctt{a}{n}$, \mtex{mvctzt\{a\}\{n\}} gets $\mvctzt{a}{n}$

\item \mtex{mequlist{}}: provided a list of equations, eg \mtex{mequlist\{x + y \&= 10 \textbackslash\textbackslash 4x + 2y \&= 30\}} gets $\mequlist{x + y &= 10 \\ 4x + 2y &= 30}$
\end{itemize}

%------------------------------------------------------------------------------%
% \end{multicols}
\end{document}