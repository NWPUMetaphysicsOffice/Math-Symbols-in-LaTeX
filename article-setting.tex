% pdf 设置区
%---------------------------------纸张大小设置---------------------------------%
\usepackage{geometry}
\geometry{left=2.5cm,right=2.5cm,top=2.5cm,bottom=2.5cm}
%------------------------------------------------------------------------------%


%----------------------------------必要库支持----------------------------------%
\usepackage{xcolor}
\usepackage{tikz}
\usepackage{layouts}
\usepackage[numbers,sort&compress]{natbib}
\usepackage{clrscode}
%------------------------------------------------------------------------------%


%--------------------------------添加书签超链接--------------------------------%
\usepackage[unicode=true,colorlinks=false,pdfborder={0 0 0}]{hyperref}
    % 在此处修改打开文件操作
    \hypersetup{
        bookmarks=true,         % show bookmarks bar?
        pdftoolbar=true,        % show Acrobat’s toolbar?
        pdfmenubar=true,        % show Acrobat’s menu?
        pdffitwindow=true,      % window fit to page when opened
        pdfstartview={FitH},    % fits the width of the page to the window
        pdfnewwindow=true,      % links in new PDF window
    }
    % 在此处添加文章基础信息
    \hypersetup{
        pdftitle={},
        pdfauthor={polossk},
        pdfsubject={},
        pdfcreator={XeLaTeX},
        pdfproducer={XeLaTeX},
        pdfkeywords={},
    }
%------------------------------------------------------------------------------%


%---------------------------------设置字体大小---------------------------------%
\usepackage{type1cm}
% 字号与行距,统一前缀s(a.k.a size)
\newcommand{\sChuhao}{\fontsize{42pt}{63pt}\selectfont}         % 初号, 1.5倍
\newcommand{\sYihao}{\fontsize{26pt}{36pt}\selectfont}          % 一号, 1.4倍
\newcommand{\sErhao}{\fontsize{22pt}{28pt}\selectfont}          % 二号, 1.25倍
\newcommand{\sXiaoer}{\fontsize{18pt}{18pt}\selectfont}         % 小二, 单倍
\newcommand{\sSanhao}{\fontsize{16pt}{24pt}\selectfont}         % 三号, 1.5倍
\newcommand{\sXiaosan}{\fontsize{15pt}{22pt}\selectfont}        % 小三, 1.5倍
\newcommand{\sSihao}{\fontsize{14pt}{21pt}\selectfont}          % 四号, 1.5倍
\newcommand{\sHalfXiaosi}{\fontsize{13pt}{19.5pt}\selectfont}   % 半小四, 1.5倍
\newcommand{\sXiaosi}{\fontsize{12pt}{14.4pt}\selectfont}       % 小四, 1.25倍
\newcommand{\sLargeWuhao}{\fontsize{11pt}{11pt}\selectfont}     % 大五, 单倍
\newcommand{\sWuhao}{\fontsize{10.5pt}{10.5pt}\selectfont}      % 五号, 单倍
\newcommand{\sXiaowu}{\fontsize{9pt}{9pt}\selectfont}           % 小五, 单倍
%------------------------------------------------------------------------------%


%---------------------------------设置中文字体---------------------------------%
\usepackage{fontspec}
\usepackage[SlantFont,BoldFont,CJKchecksingle]{xeCJK}
\usepackage{CJKnumb}

% 使用 Adobe 字体
\newcommand\adobeSog{Adobe Song Std}
\newcommand\adobeHei{Adobe Heiti Std}
\newcommand\adobeKai{Adobe Kaiti Std}
\newcommand\adobeFag{Adobe Fangsong Std}
\newcommand\codeFont{Consolas}
% 设置字体
\defaultfontfeatures{Mapping=tex-text}
\setCJKmainfont[ItalicFont=\adobeKai, BoldFont=\adobeHei]{\adobeSog}
\setCJKsansfont[ItalicFont=\adobeKai, BoldFont=\adobeHei]{\adobeSog}
\setCJKmonofont{\codeFont}
\setmonofont{\codeFont}
% 设置字体族
\setCJKfamilyfont{song}{\adobeSog}      % 宋体  
\setCJKfamilyfont{hei}{\adobeHei}       % 黑体  
\setCJKfamilyfont{kai}{\adobeKai}       % 楷体  
\setCJKfamilyfont{fang}{\adobeFag}      % 仿宋体
% 用于页眉学校名,特殊字体,powerby https://github.com/ecomfe/fonteditor
\setCJKfamilyfont{nwpu}{nwpuname}
% 新建字体命令,统一前缀f(a.k.a font)
\newcommand{\fSong}{\CJKfamily{song}}
\newcommand{\fHei}{\CJKfamily{hei}}
\newcommand{\fFang}{\CJKfamily{fang}}
\newcommand{\fKai}{\CJKfamily{kai}}
\newcommand{\fNWPU}{\CJKfamily{nwpu}}
%------------------------------------------------------------------------------%


%------------------------------添加插图与表格控制------------------------------%
\usepackage{graphicx}
\usepackage[font=small,labelsep=quad]{caption}
\usepackage{wrapfig}
\usepackage{multirow,makecell}
\usepackage{longtable}
\usepackage{booktabs}
\usepackage{tabularx}
\usepackage{setspace}
%------------------------------------------------------------------------------%


%---------------------------------添加列表控制---------------------------------%
\usepackage{enumerate}
\usepackage{enumitem}
%------------------------------------------------------------------------------%


%---------------------------------设置引用格式---------------------------------%
\renewcommand\figureautorefname{图}
\renewcommand\tableautorefname{表}
\renewcommand\equationautorefname{式}
\newcommand\myreference[1]{[\ref{#1}]}
\newcommand\eqrefe[1]{式(\ref{#1})}
\renewcommand\theequation{\thechapter.\arabic{equation}}
% 增加 \ucite 命令使显示的引用为上标形式
\newcommand{\ucite}[1]{$^{\mbox{\scriptsize \cite{#1}}}$}
%------------------------------------------------------------------------------%


%--------------------------设置中文段落缩进与正文版式--------------------------%
\XeTeXlinebreaklocale "zh"       %使用中文的换行风格
\XeTeXlinebreakskip = 0pt plus 1pt    %调整换行逻辑的弹性大小
%------------------------------------------------------------------------------%

%---------------------------------设置页眉页脚---------------------------------%
\usepackage{fancyhdr}
\usepackage{fancyref}
\pagestyle{fancy}
% \chead{{\fNWPU 西北工业大学} \fSong {\title}}
\lfoot{}
\cfoot{\thepage}
\rfoot{}
\renewcommand{\headrulewidth}{0.4pt}
\renewcommand{\headwidth}{\textwidth}
\renewcommand{\footrulewidth}{0pt}
%------------------------------------------------------------------------------%


%-------------------------------数学特殊符号控制-------------------------------%
% math-symbols.sty
% by polossk, version 2.1.0.0402
% last update 2019-04-02
% github.com/polossk/Math-Symbols-in-LaTeX
\ProvidesPackage{math-symbols}[2019/04/02 v2.1.0.0402 Math-Symbols-in-LaTeX]
%------------------------------------------------------------------------------%
%-Requare-Packages-------------------------------------------------------------%
%------------------------------------------------------------------------------%
\RequirePackage{bm}
\RequirePackage{amsmath}
\RequirePackage{amssymb}
\RequirePackage{amsfonts}
\RequirePackage{mathrsfs}
\RequirePackage{mathtools}
%------------------------------------------------------------------------------%
%-Math-Fonts-------------------------------------------------------------------%
%------------------------------------------------------------------------------%
\def\mathbi#1{\textbf{\em #1}}
% \mathcal{A}:  Caligraphic letters
% \mathbb{A}:   Mathbb letters
% \mathfrak{A}: Mathfrak letters
% \mathsf{A}:   Math Sans serif letters
% \mathbf{A}:   Math bold letters
% \mathbi{A}:   Math bold italic letters
%------------------------------------------------------------------------------%
%-Constants-and-Useful-Symbols-------------------------------------------------%
%------------------------------------------------------------------------------%
\newcommand\mi{{\mathrm i}}                                % constant i
\newcommand\mj{{\mathrm j}}                                % constant j
\newcommand\me{{\mathrm e}}                                % constant e
\newcommand\mdeg{{^\circ}}                                 % symbol degree
\newcommand\mnatr{{\mathbb N}}                             % natural num set N
\newcommand\mintg{{\mathbb Z}}                             % integer set Z
\newcommand\mrato{{\mathbb Q}}                             % rational set Q
\newcommand\mreal{{\mathbb R}}                             % real num set R
\newcommand\mcmpx{{\mathbb C}}                             % complex set C
\newcommand\mhilb{{\mathbb H}}                             % hilbert set H
\newcommand\mcond{{\mathrm{Cond.}}}                        % condition symbol
\newcommand\mconst{{\mathrm{const}}}                       % constant symbol
\newcommand\mscon[1]{C{#1}}                                % continuous fspace
\newcommand\mscab{{\mscon{[a, b]}}}                        % continuous fspace
\newcommand\mslbg[2][I]{{L^{#2}({#1})}}                    % lebesgue fspace
\newcommand\mssbl[2][I]{{H^{#2}({#1})}}                    % sobolev fspace
%------------------------------------------------------------------------------%
%-[M]ath-[V]ector-Defination---------------------------------------------------%
%------------------------------------------------------------------------------%
\newcommand\mva{{\bm a}}                                   % vector a
\newcommand\mvb{{\bm b}}                                   % vector b
\newcommand\mvc{{\bm c}}                                   % vector c
\newcommand\mvd{{\bm d}}                                   % vector d
\newcommand\mve{{\bm e}}                                   % vector e
\newcommand\mvf{{\bm f}}                                   % vector f
\newcommand\mvg{{\bm g}}                                   % vector g
\newcommand\mvh{{\bm h}}                                   % vector h
\newcommand\mvi{{\bm i}}                                   % vector i
\newcommand\mvj{{\bm j}}                                   % vector j
\newcommand\mvk{{\bm k}}                                   % vector k
\newcommand\mvl{{\bm l}}                                   % vector l
\newcommand\mvm{{\bm m}}                                   % vector m
\newcommand\mvn{{\bm n}}                                   % vector n
\newcommand\mvo{{\bm o}}                                   % vector o
\newcommand\mvp{{\bm p}}                                   % vector p
\newcommand\mvq{{\bm q}}                                   % vector q
\newcommand\mvr{{\bm r}}                                   % vector r
\newcommand\mvs{{\bm s}}                                   % vector s
\newcommand\mvt{{\bm t}}                                   % vector t
\newcommand\mvu{{\bm u}}                                   % vector u
\newcommand\mvv{{\bm v}}                                   % vector v
\newcommand\mvw{{\bm w}}                                   % vector w
\newcommand\mvx{{\bm x}}                                   % vector x
\newcommand\mvy{{\bm y}}                                   % vector y
\newcommand\mvz{{\bm z}}                                   % vector z
\newcommand\mvalpha{{\bm \alpha}}                          % vector alpha
\newcommand\mvbeta{{\bm \beta}}                            % vector beta
\newcommand\mvgamma{{\bm \gamma}}                          % vector gamma
\newcommand\mvdelta{{\bm \delta}}                          % vector delta
\newcommand\mvepsilon{{\bm \epsilon}}                      % vector epsilon
\newcommand\mvzeta{{\bm \zeta}}                            % vector zeta
\newcommand\mveta{{\bm \eta}}                              % vector eta
\newcommand\mvtheta{{\bm \theta}}                          % vector theta
\newcommand\mviota{{\bm \iota}}                            % vector iota
\newcommand\mvkappa{{\bm \kappa}}                          % vector kappa
\newcommand\mvlambda{{\bm \lambda}}                        % vector lambda
\newcommand\mvmu{{\bm \mu}}                                % vector mu
\newcommand\mvnu{{\bm \nu}}                                % vector nu
\newcommand\mvxi{{\bm \xi}}                                % vector xi
\newcommand\mvpi{{\bm \pi}}                                % vector pi
\newcommand\mvrho{{\bm \rho}}                              % vector rho
\newcommand\mvsigma{{\bm \sigma}}                          % vector sigma
\newcommand\mvtau{{\bm \tau}}                              % vector tau
\newcommand\mvupsilon{{\bm \upsilon}}                      % vector upsilon
\newcommand\mvphi{{\bm \phi}}                              % vector phi
\newcommand\mvchi{{\bm \chi}}                              % vector chi
\newcommand\mvpsi{{\bm \psi}}                              % vector psi
\newcommand\mvomega{{\bm \omega}}                          % vector omega
\newcommand\mvvarepsilon{{\bm \varepsilon}}                % vector varepsilon
\newcommand\mvvarkappa{{\bm \varkappa}}                    % vector varkappa
\newcommand\mvvarphi{{\bm \varphi}}                        % vector varphi
\newcommand\mvvarpi{{\bm \varpi}}                          % vector varpi
\newcommand\mvvarrho{{\bm \varrho}}                        % vector varrho
\newcommand\mvvartheta{{\bm \vartheta}}                    % vector vartheta
%------------------------------------------------------------------------------%
%-[M]ath-[M]atrix-Defination---------------------------------------------------%
%------------------------------------------------------------------------------%
\newcommand\mma{{\mathbf A}}                               % matrix A
\newcommand\mmb{{\mathbf B}}                               % matrix B
\newcommand\mmc{{\mathbf C}}                               % matrix C
\newcommand\mmd{{\mathbf D}}                               % matrix D
\newcommand\mme{{\mathbf E}}                               % matrix E
\newcommand\mmf{{\mathbf F}}                               % matrix F
\newcommand\mmg{{\mathbf G}}                               % matrix G
\newcommand\mmh{{\mathbf H}}                               % matrix H
\newcommand\mmi{{\mathbf I}}                               % matrix I
\newcommand\mmj{{\mathbf J}}                               % matrix J
\newcommand\mmk{{\mathbf K}}                               % matrix K
\newcommand\mml{{\mathbf L}}                               % matrix L
\newcommand\mmm{{\mathbf M}}                               % matrix M
\newcommand\mmn{{\mathbf N}}                               % matrix N
\newcommand\mmo{{\mathbf O}}                               % matrix O
\newcommand\mmp{{\mathbf P}}                               % matrix P
\newcommand\mmq{{\mathbf Q}}                               % matrix Q
\newcommand\mmr{{\mathbf R}}                               % matrix R
\newcommand\mms{{\mathbf S}}                               % matrix S
\newcommand\mmt{{\mathbf T}}                               % matrix T
\newcommand\mmu{{\mathbf U}}                               % matrix U
\newcommand\mmv{{\mathbf V}}                               % matrix V
\newcommand\mmw{{\mathbf W}}                               % matrix W
\newcommand\mmx{{\mathbf X}}                               % matrix X
\newcommand\mmy{{\mathbf Y}}                               % matrix Y
\newcommand\mmz{{\mathbf Z}}                               % matrix Z
\newcommand\mmgamma{{\bm \Gamma}}                          % matrix Gamma
\newcommand\mmdelta{{\bm \Delta}}                          % matrix Delta
\newcommand\mmtheta{{\bm \Theta}}                          % matrix Theta
\newcommand\mmlambda{{\bm \Lambda}}                        % matrix Lambda
\newcommand\mmxi{{\bm \Xi}}                                % matrix Xi
\newcommand\mmpi{{\bm \Pi}}                                % matrix Pi
\newcommand\mmsigma{{\bm \Sigma}}                          % matrix Sigma
\newcommand\mmupsilon{{\bm \Upsilon}}                      % matrix Upsilon
\newcommand\mmphi{{\bm \Phi}}                              % matrix Phi
\newcommand\mmpsi{{\bm \Psi}}                              % matrix Psi
\newcommand\mmomega{{\bm \Omega}}                          % matrix Omega
%------------------------------------------------------------------------------%
%-[M]ath-[T]ensor-Defination---------------------------------------------------%
%------------------------------------------------------------------------------%
\newcommand\mta{{\mathbf A}}                               % tensor A
\newcommand\mtb{{\mathbf B}}                               % tensor B
\newcommand\mtc{{\mathbf C}}                               % tensor C
\newcommand\mtd{{\mathbf D}}                               % tensor D
\newcommand\mte{{\mathbf E}}                               % tensor E
\newcommand\mtf{{\mathbf F}}                               % tensor F
\newcommand\mtg{{\mathbf G}}                               % tensor G
\newcommand\mth{{\mathbf H}}                               % tensor H
\newcommand\mti{{\mathbf I}}                               % tensor I
\newcommand\mtj{{\mathbf J}}                               % tensor J
\newcommand\mtk{{\mathbf K}}                               % tensor K
\newcommand\mtl{{\mathbf L}}                               % tensor L
\newcommand\mtm{{\mathbf M}}                               % tensor M
\newcommand\mtn{{\mathbf N}}                               % tensor N
\newcommand\mto{{\mathbf O}}                               % tensor O
\newcommand\mtp{{\mathbf P}}                               % tensor P
\newcommand\mtq{{\mathbf Q}}                               % tensor Q
\newcommand\mtr{{\mathbf R}}                               % tensor R
\newcommand\mts{{\mathbf S}}                               % tensor S
\newcommand\mtt{{\mathbf T}}                               % tensor T
\newcommand\mtu{{\mathbf U}}                               % tensor U
\newcommand\mtv{{\mathbf V}}                               % tensor V
\newcommand\mtw{{\mathbf W}}                               % tensor W
\newcommand\mtx{{\mathbf X}}                               % tensor X
\newcommand\mty{{\mathbf Y}}                               % tensor Y
\newcommand\mtz{{\mathbf Z}}                               % tensor Z
\newcommand\mtgamma{{\bm \Gamma}}                          % tensor Gamma
\newcommand\mtdelta{{\bm \Delta}}                          % tensor Delta
\newcommand\mttheta{{\bm \Theta}}                          % tensor Theta
\newcommand\mtlambda{{\bm \Lambda}}                        % tensor Lambda
\newcommand\mtxi{{\bm \Xi}}                                % tensor Xi
\newcommand\mtpi{{\bm \Pi}}                                % tensor Pi
\newcommand\mtsigma{{\bm \Sigma}}                          % tensor Sigma
\newcommand\mtupsilon{{\bm \Upsilon}}                      % tensor Upsilon
\newcommand\mtphi{{\bm \Phi}}                              % tensor Phi
\newcommand\mtpsi{{\bm \Psi}}                              % tensor Psi
\newcommand\mtomega{{\bm \Omega}}                          % tensor Omega
%------------------------------------------------------------------------------%
%-[M]ath-[V]ector-Defination-[T]ransposed--------------------------------------%
%------------------------------------------------------------------------------%
\newcommand\mmat{{{\mathbf A}^T}}                          % matrix A^T
\newcommand\mmbt{{{\mathbf B}^T}}                          % matrix B^T
\newcommand\mmct{{{\mathbf C}^T}}                          % matrix C^T
\newcommand\mmdt{{{\mathbf D}^T}}                          % matrix D^T
\newcommand\mmet{{{\mathbf E}^T}}                          % matrix E^T
\newcommand\mmft{{{\mathbf F}^T}}                          % matrix F^T
\newcommand\mmgt{{{\mathbf G}^T}}                          % matrix G^T
\newcommand\mmht{{{\mathbf H}^T}}                          % matrix H^T
\newcommand\mmit{{{\mathbf I}^T}}                          % matrix I^T
\newcommand\mmjt{{{\mathbf J}^T}}                          % matrix J^T
\newcommand\mmkt{{{\mathbf K}^T}}                          % matrix K^T
\newcommand\mmlt{{{\mathbf L}^T}}                          % matrix L^T
\newcommand\mmmt{{{\mathbf M}^T}}                          % matrix M^T
\newcommand\mmnt{{{\mathbf N}^T}}                          % matrix N^T
\newcommand\mmot{{{\mathbf O}^T}}                          % matrix O^T
\newcommand\mmpt{{{\mathbf P}^T}}                          % matrix P^T
\newcommand\mmqt{{{\mathbf Q}^T}}                          % matrix Q^T
\newcommand\mmrt{{{\mathbf R}^T}}                          % matrix R^T
\newcommand\mmst{{{\mathbf S}^T}}                          % matrix S^T
\newcommand\mmtt{{{\mathbf T}^T}}                          % matrix T^T
\newcommand\mmut{{{\mathbf U}^T}}                          % matrix U^T
\newcommand\mmvt{{{\mathbf V}^T}}                          % matrix V^T
\newcommand\mmwt{{{\mathbf W}^T}}                          % matrix W^T
\newcommand\mmxt{{{\mathbf X}^T}}                          % matrix X^T
\newcommand\mmyt{{{\mathbf Y}^T}}                          % matrix Y^T
\newcommand\mmzt{{{\mathbf Z}^T}}                          % matrix Z^T
\newcommand\mmgammat{{{\bm \Gamma}^T}}                     % matrix Gamma^T
\newcommand\mmdeltat{{{\bm \Delta}^T}}                     % matrix Delta^T
\newcommand\mmthetat{{{\bm \Theta}^T}}                     % matrix Theta^T
\newcommand\mmlambdat{{{\bm \Lambda}^T}}                   % matrix Lambda^T
\newcommand\mmxit{{{\bm \Xi}^T}}                           % matrix Xi^T
\newcommand\mmpit{{{\bm \Pi}^T}}                           % matrix Pi^T
\newcommand\mmsigmat{{{\bm \Sigma}^T}}                     % matrix Sigma^T
\newcommand\mmupsilont{{{\bm \Upsilon}^T}}                 % matrix Upsilon^T
\newcommand\mmphit{{{\bm \Phi}^T}}                         % matrix Phi^T
\newcommand\mmpsit{{{\bm \Psi}^T}}                         % matrix Psi^T
\newcommand\mmomegat{{{\bm \Omega}^T}}                     % matrix Omega^T
%------------------------------------------------------------------------------%
\newcommand\mvzero{{\bm 0}}                                % vector 0
\newcommand\mvone{{\bm 1}}                                 % vector 1
\newcommand\mmzero{{\mathbf 0}}                            % matrix 0
\newcommand\mmone{{\mathbf 1}}                             % matrix 1
\newcommand\mtzero{{\mathbf 0}}                            % tensor 0
\newcommand\mtone{{\mathbf 1}}                             % tensor 1
%------------------------------------------------------------------------------%
%-Useful-Operators-------------------------------------------------------------%
%------------------------------------------------------------------------------%
\DeclareMathOperator\diff{d\!}                             % operator diff
\DeclareMathOperator\Diff{D}                               % operator Diff
\DeclareMathOperator\Expect{E}                             % operator Expect
\DeclareMathOperator\diag{diag}                            % operator diag
\DeclareMathOperator\eig{eig}                              % operator eig
\DeclareMathOperator\tr{tr}                                % operator tr
\DeclareMathOperator\lcm{lcm}                              % operator lcm
\DeclareMathOperator\rand{rand}                            % operator rand
\DeclareMathOperator\mean{mean}                            % operator mean
\DeclareMathOperator\var{var}                              % operator var
\DeclareMathOperator\corr{corr}                            % operator corr
\DeclareMathOperator\conv{conv}                            % operator conv
\DeclareMathOperator\card{card}                            % operator card
\DeclareMathOperator*{\argmin}{argmin}                     % operator argmin
\DeclareMathOperator*{\argmax}{argmax}                     % operator argmax
\DeclareMathOperator*{\argopt}{argopt}                     % operator argopt
\DeclareMathOperator*{\dist}{dist}                         % operator dist
\DeclareMathOperator\rot{rot}                              % operator rot
\DeclareMathOperator\curl{curl}                            % operator curl
\DeclareMathOperator\divergence{div}                       % operator div
%------------------------------------------------------------------------------%
%-Useful-Functions-------------------------------------------------------------%
%------------------------------------------------------------------------------%
\newcommand\abs[1]{{\lvert {#1} \rvert}}                   % operator abs
\newcommand\norm[1]{{\| {#1} \|}}                          % operator norm
\newcommand\normlp[2]{{\norm{#1}}_{#2}}                    % operator Lp-norm
%------------------------------------------------------------------------------%
%-Auto-Line-Break-Config-------------------------------------------------------%
%------------------------------------------------------------------------------%
\allowdisplaybreaks[4]
%------------------------------------------------------------------------------%
%-Useful-Aliases-and-Generators------------------------------------------------%
%------------------------------------------------------------------------------%
\newcommand\fracdiff[2]{{\frac{\diff {#1}}{\diff {#2}}}}
\newcommand\fracdiffs[1]{{\fracdiff{}{#1}}}
\newcommand\dfracdiff[2]{{\dfrac{\diff {#1}}{\diff {#2}}}}
\newcommand\dfracdiffs[1]{{\dfracdiff{}{#1}}}
\newcommand\fracpartial[2]{{\frac{\partial {#1}}{\partial {#2}}}}
\newcommand\fracpartials[1]{{\fracpartial{}{#1}}}
\newcommand\dfracpartial[2]{{\dfrac{\partial {#1}}{\partial {#2}}}}
\newcommand\dfracpartials[1]{{\dfracpartial{}{#1}}}


\newcommand\mfwhen[2]{{\left.{#1} \right|}_{#2}}

\newcommand\mclosure[1]{{\left( {#1} \right)}}
\newcommand\mclosuresquare[1]{{\left[ {#1} \right]}}
\newcommand\mclosurebrace[1]{{\left\{ {#1} \right\}}}

\newcommand\mvct[2]{{ ({#1}_1, {#1}_2, \ldots, {#1}_{#2}) }}
\newcommand\mvctz[2]{{ ({#1}_0, {#1}_1, \ldots, {#1}_{#2}) }}
\newcommand\mvctt[2]{{ \mvct{#1}{#2}^T }}
\newcommand\mvctzt[2]{{ \mvctz{#1}{#2}^T }}

\newcommand\mequlist[1]{{\left\{ \begin{aligned} #1 \end{aligned} \right.}}
\newenvironment{equlist}{\left\{ \begin{aligned} }{ \end{aligned} \right.}
%------------------------------------------------------------------------------%
%------------------------------------------------------------------------------%
%------------------------------------------------------------------------------%
% 这是所有常用的数学符号的 style file,不能单独编译。
% build by ruby and python scripts

%------------------------------------------------------------------------------%


%---------------------------------添加列表控制---------------------------------%
\usepackage{enumerate}
\numberwithin{equation}{section}
\renewcommand\theequation{\thesection.\arabic{equation}}
%------------------------------------------------------------------------------%


%---------------------------------添加代码控制---------------------------------%
\usepackage{listings}
\lstset{
    basicstyle=\footnotesize\ttfamily,
    %numbers=left,
    %numberstyle=\tiny,
    %numbersep=5pt,
    tabsize=4,
    extendedchars=true,
    breaklines=true,
    keywordstyle=\color{blue},
    numberstyle=\color{purple},
    commentstyle=\color{olive},
    stringstyle=\color{orange}\ttfamily,
    showspaces=false,
    showtabs=false,
    framexrightmargin=5pt,
    framexbottommargin=4pt,
    showstringspaces=false,
    escapeinside=`', %逃逸字符(1左面的键),用于显示中文
}
\renewcommand{\lstlistingname}{CODE}
\lstloadlanguages{% Check Dokumentation for further languages, page 12
    Pascal, C++, Java, Ruby, Python, Matlab, R
}
%------------------------------------------------------------------------------%

\endinput
%这是简单的 article 的导言区设置,不能单独编译。